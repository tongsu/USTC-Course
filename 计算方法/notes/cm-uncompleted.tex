\documentclass[a4paper,12pt]{article}
\def\pgfsysdriver{pgfsys-dvipdfm.def}

\usepackage[no-math]{fontspec}
\usepackage{indentfirst}
\usepackage{xunicode}% provides unicode character macros
\usepackage{xltxtra} % provides some fixes/extras
\usepackage{tikz}

\XeTeXlinebreaklocale "zh"
\XeTeXlinebreakskip = 0pt plus 1pt minus 0.1pt

\usepackage{amsmath}
\usepackage{bm}
\usepackage{mathpazo}
\usepackage{color}
\usepackage{mdwlist}
\usepackage{paralist}
\usepackage{enumerate}
\usepackage{url}
\usepackage{latexsym}
\usepackage{hyperref}
\usepackage{ulem}
\hypersetup{
    colorlinks,
    citecolor=blue,
    filecolor=black,
    linkcolor=black,
    urlcolor=brown
}


\setlength{\parskip}{5pt}
%\newfontfamily\hei{"黑体"}

%\setmainfont{msyhl.ttc}
\setmainfont[BoldFont={Deng.ttf}]{Dengl.ttf}
%\setmainfont[BoldFont={timesbd.ttf},ItalicFont={timesi.ttf}]{times.ttf}
%\setmathfont{BRADHITC.TTF}


\begin{document}
\title{Notes of Computational Method}
\author{Michael Zhu}
\date{2016.6.8}
\maketitle
\tableofcontents
\newpage
\setcounter{section}{-1}
\section{绪论}
\paragraph{绝对误差} $e=x^*-x$
\subparagraph{绝对误差限} $|e|\le\epsilon$
\paragraph{相对误差} $e_r=\frac{x^*-x}{x^*}$
\subparagraph{相对误差限} $|e_r|\le\epsilon_r$
\paragraph{产生误差的主要原因} 原始误差,截断误差,舍入误差
\paragraph{有效位数} 当 $x$ 的误差限为某一位的半个单位,则这一位到第一个非零位的位数称为 $x$ 的有效位数。
\paragraph{向量范数($L_p$)} $\|\mathbf{X}\|_p=\left(\sum_{i=1}^{n}|x_i|^p\right)^{1/p},\ \ 1\le p\le\infty.$
\subparagraph{$p=1$} $\|\mathbf{X}\|_1=\sum_{i=1}^{n}|x_i|$
\subparagraph{$p=2$} $\|\mathbf{X}\|_2=\sqrt{\sum_{i=1}^{n}|x_i|^2}$
\subparagraph{$p=\infty$} $\|\mathbf{X}\|_\infty=\underset{1\le i\le n}{\mathrm{max}}\{|x_i|\}$
\paragraph{矩阵范数} $\|\mathsf{A}\|=\underset{\underset{x\not=0}{x\in\mathbf{R}^n}}{\mathrm{sup}}\frac{\|\mathsf{A}\mathbf{X}\|}{\|\mathbf{X}\|}$\quad 或 \quad $\|\mathsf{A}\|=\underset{\underset{\|x\|=1}{x\in\mathbf{R}^n}}{\mathrm{sup}}\|\mathsf{A}\mathbf{X}\|$
\subparagraph{$p=1$} $\|\mathsf{A}\|_1=\underset{1\le j\le n}{\mathrm{max}}\sum_{i=1}^{n}|a_{i\ j}|$
\subparagraph{$p=\infty$} $\|\mathsf{A}\|_\infty=\underset{1\le i\le n}{\mathrm{max}}\sum_{j=1}^{n}|a_{i\ j}|$
\subparagraph{$p=2$} $\|\mathsf{A}\|_2=(\rho(\mathsf{A}^TA))^{1/2}$
\subparagraph{谱半径} $\rho(\mathsf{A})=\underset{i}{\mathrm{max}}\{|\lambda_i|\}$
\subparagraph{Euclid/Schur} $\|\mathsf{A}\|_E=(\sum_{i=1}^{n}\sum_{j=1}^{n}|a_{ij}|^2)^{1/2}$ 
\paragraph{特征值与范数} $|\lambda|\le\|\mathsf{A}\|$,\ \ $\rho(\mathsf{A})\le\|\mathsf{A}\|.$

\section{插值}
\subsection{Lagrange 插值}
\paragraph{基函数}
\[
l_i(x)=\underset{\underset{j\not=i}{0\le j\le n}}{\prod} \frac{x-x_j}{x_i-x_j}
\]
\paragraph{插值多项式}
\[
L_n(x)=\sum_{i=0}^{n}l_i(x)f(x_i)
\]
\paragraph{误差}
\[
R_n(x)=\frac{f^{(n+1)}(\xi)}{(n+1)!}\prod_{i=0}^{n}(x-x_i),\quad \xi\in[a,b]
\]
\subsection{Newton 插值}
\paragraph{差商}
\subparagraph{一阶差商} $f[x_0,x_1]=\frac{f(x_1)-f(x_0)}{x_1-x_0}$
\subparagraph{$k$ 阶差商} $f[x_0,x_1,\cdots,x_n]=\frac{f[x_1,x_2,\cdots,x_k]-f[x_0,x_1,\cdots,x_{k-1}]}{x_k-x_0}$
\paragraph{差商的性质}
\subparagraph{(1)} 
\[
f[x_0,x_1,\cdots,x_n]=\sum_{i=0}^{k}\frac{f(x_i)}{\underset{\underset{j\not=i}{0\le j\le n}}{\prod}(x_i-x_j)}
\]
\subparagraph{(2)} 差商与节点的顺序无关。 
\paragraph{插值多项式}
\[
N(x)=f(x_0)+(x-x_0)f[x_0,x_1]+\cdots+(x-x_0)(x-x_1)\cdots(x-x_{n-1})f[x_0,x_1,\cdots,x_n]
\]
\[
N(x)=f(x_0)+\sum_{k=1}^{n}f[x_0,\cdots,x_k](x-x_0)\cdots(x-x^{k-1})
\]
\paragraph{误差}
\[
R_n(x)=\frac{f^{(n+1)}(\xi)}{(n+1)!}\prod_{i=0}^{n}(x-x_i)=f[x,x_0,\cdots,x_n]\prod_{i=0}^{n}(x-x_i)
\]
和 Lagrange 插值多项式的误差完全相等。
\subsection{Hermite 插值}
\paragraph{构造基函数}
\paragraph{利用差商构造}
\subsection{分段插值}
\paragraph{插值函数}
\[
p_i(x)=\frac{x-x_{i+1}}{x_i-x_{i+1}}f(x_i)+\frac{x-x_{i}}{x_{i+1}-x_{i}}f(x_{i+1}),\quad x_i\le x\le x_{i+1}
\]
\subsection{三次样条函数}

\section{数值微分和数值积分}
\subsection{数值微分}
\subsubsection{差商和数值微分}
\paragraph{差商}
\subparagraph{向前差商}
\[
f'(x_0)\approx\frac{f(x_0+h)-f(x_0)}{h}
\]
\[
R(x)=-\frac{h}{2}f''(\xi)=O(h)
\]
\subparagraph{向后差商}
\[
f'(x_0)\approx\frac{f(x_0)-f(x_0-h)}{h}
\]
\[
R(x)=\frac{h}{2}f''(\xi)=O(h)
\]
\subparagraph{中心差商}
\[
f'(x_0)\approx\frac{f(x_0+h)-f(x_0-h)}{2h}
\]
\[
R(x)=-\frac{h^2}{6}f'''(\xi)=O(h^2)
\]
\subsubsection{插值型数值微分}
\[
f(x)\approx L_n(x)=\sum_{i=0}^{n}l_i(x)f(x_i)
\]
\[
f'(x)\approx L'_n(x)=\sum_{i=0}^{n}l'_i(x)f(x_i)
\]
\paragraph{误差项}
\[
R(x)=\frac{\mathrm{d}}{\mathrm{d}x}\left(\frac{f^{(n+1)}(\xi)}{(n+1)!}\prod_{i=0}^{n}(x-x_i)\right)
\]
\paragraph{$x=x_j$}
\[
f'(x_j)=\sum_{i=0}^{n}l'_i(x_j)f(x_i)
\]
\[
R(x_j)=\underset{\underset{i\not=j}{i=0}}{\prod^n}(x_j-x_i)\frac{f^{(n+1)}(\xi)}{(n+1)!}
\]
\subsection{数值积分}
\paragraph{代数精度}
\[
E_n(x^k)=I(x^k)-I_n(x^k)=0,\ \ k=0,1,\cdots,m
\]
而
\[
E_n(x^{m+1})\not=0,
\]
则称 $I_n(f)$ 具有 $m$ 阶代数精度。具有 $m$ 阶代数精度时,对于任意不高于 $m$ 次的多项式 $f(x)$ 都有 $I(f)=I_n(f)$.
\subsubsection{插值型数值微分}
\[
\int_a^b f(x)\mathrm{d}x\approx\int_a^b L_n(x)\mathrm{d}x=\sum_{i=0}^n\left[\int_a^b l_i(x)\mathrm{d}x\right]f(x_i)
\]
\paragraph{误差}
\[
E_n(f)=\int_a^bf[x_0,x_1,\cdots,x_n,x]\prod_{i=0}^{n}(x-x_i)\mathrm{d}x
\]
\subsubsection{Newton-Cotes 积分}
取等距节点,亦即对区间做等距剖分。\par
$n$ 为偶数时具有 $n+1$ 阶代数精度,$n$ 为奇数时具有 $n$ 阶代数精度。
\paragraph{梯形积分}
\[
T(f)=\frac{b-a}{2}[f(a)+f(b)]
\]
\[
E_1(x)=-\frac{f''(\eta)}{12}(b-a)^3
\]
具有 $1$ 阶代数精度。
\paragraph{Simpson 积分}
\[
S(f)=\frac{b-a}{6}[f(a)+4f(\frac{a+b}{2})+f(b)]
\]
\[
E_2(f)=-\frac{f^{(4)}(\eta)}{2880}(b-a)^5
\]
具有 $3$ 阶代数精度。
\subsection{复化数值积分}
\subsubsection{复化梯形积分}
\[
T(h)=T_n(f)=h\left[\frac{1}{2}f(a)+\sum_{i=0}^{n-1}f(a+ih)+\frac{1}{2}f(b)\right]
\]
\paragraph{截断误差}
\[
E_n(f)=-\frac{(b-a)^3}{12n^2}f''(\xi)\sim O(h^2)
\]
\subsubsection{复化 Simpson 积分}
\[
S_n(f)=\frac{h}{3}\left[f(a)+4\sum_{i=0}^{m-1}f(x_{2i+1})+2\sum_{i=1}^{m-1}f(x_{2i})+f(b)\right]
\]
\paragraph{截断误差}
\[
E_n(f)=-\frac{(b-a)^5}{180n^4}f^{(4)}(\xi)\sim O(h^4)
\]
\subsubsection{Romberg 算法}
\[
R_{k,j}=R_{k,j-1}+\frac{R_{k,j-1}-R_{k-1,j-1}}{4^{j-1}-1}
\]
\subsection{重积分计算}
\subsubsection{复化梯形积分}
$$\int_a^b\int_c^df(x,y)\mathrm{d}x\mathrm{d}y\approx hk\sum_i\sum_jc_{ij}f(x_i,y_j),\ \ c_{ij}=\Bigg\{
\begin{array}{ll}
\frac{1}{4},&\text{角点}\\
\frac{1}{2},&\text{边点}\\
1,&\text{内点}\\
\end{array}$$
\subsection{Gauss 型积分}
$2n-1$ 阶代数精度, $n$ 个节点的最高代数精度。
\[
G_n(f)=\frac{(b-a)}{2}\sum_{i=1}^{n}\alpha_i^{(n)}f\left(\frac{(a+b)+(b-a)x_i^{(n)}}{2}\right)
\]
\section{曲线拟合的最小二乘法}
\subsection{线性拟合与二次拟合}
\subsubsection{线性拟合}
$$\left(\begin{array}{cc}
m&\sum x_i\\
\sum x_i&\sum x_i^2
\end{array}\right)
\left(\begin{array}{c}
a\\
b
\end{array}\right)=
\left(\begin{array}{c}
\sum y_i\\
\sum x_iy_i
\end{array}\right)$$
\subsubsection{二次拟合}
$$\left(\begin{array}{ccc}
m&\sum x_i&\sum x_i^2\\
\sum x_i&\sum x_i^2&\sum x_i^3\\
\sum x_i^2&\sum x_i^3&\sum x_i^4
\end{array}\right)
\left(\begin{array}{c}
a_0\\
a_1\\
a_2
\end{array}\right)=
\left(\begin{array}{c}
\sum y_i\\
\sum x_iy_i\\
\sum x_i^2y_i
\end{array}\right)$$\par
更高次如法炮制即可。\par
非多项式可以做代换,但代换后不一定满足平方误差极小。
\subsection{解矛盾方程组}
求 $\|\mathsf{A}\alpha-Y\|^2$ 的极小问题\textbf{等价}于解方程组
\[
\mathsf{A}^T\mathsf{A}\alpha=\mathsf{A}^TY.
\]
\section{非线性方程求根}
\subsection{二分法}
算法简单,但是有局限:只能算出其中一个、必须要求 $f(a)f(b)<0$ 、只能计算实根。
\subsection{迭代法}
\paragraph{基本步骤}
\subparagraph{(1)}构造等价形式 $f(x)=0\Leftrightarrow x=\phi(x)$
\subparagraph{(2)}取合适初值 $x_0$ 构造迭代序列 $x_{k+1}=\phi(x_k)$
\subparagraph{(3)}若极限不存在,可以考虑更换初值或者迭代格式。
\paragraph{压缩映射定理}
$\phi(x)\in C^1[a,b]$ 满足 
\[
a\le\phi(x)\le b,x\in[a,b]
\] 
及 
\[
\exists\, 0<L<1,\forall x\in[a,b], |\phi'(x)|\le L,
\]
则 $\phi(x)$ 不动点唯一,且对于 $x_0\in[a,b]$ 的迭代序列有误差估计
\[
|x^*-x_k|\le\frac{L^k}{1-L}|x_1-x_0|.
\]
\subsection{Newton 法}
\[
x_{k+1}=x_k-\frac{f(x_k)}{f'(x_k)}
\]\par
对于单重根一定收敛,为二阶方法。\par
对于 $p$ 重根,为一阶迭代方法,但可取以下格式,则仍为二阶方法
\[
x_{k+1}=x_k-p\frac{f(x_k)}{f'(x_k)}
\]
\subsection{弦截法}
用差商代替导数
\[
x_{k+1}=x_k-f(x_k)\frac{x_k-x_{k-1}}{f(x_k)-f(x_{k-1})}
\]\par
为 $1.618$ 阶迭代方法。
\subsection{方程组的 Newton 法}
\[
\mathbf{X}_{k+1}=\mathbf{X}_{k}-\mathsf{J}^{-1}(\mathbf{X}_{k})F(\mathbf{X}_{k})
\]\par
若
\[
\|\mathsf{J}(\mathbf{X})\|_\infty\le L<1,
\]
则迭代收敛。
\section{线性方程组--直接法}
\subsection{消元法}
\subsubsection{Gauss 消元法}
系数矩阵化为上三角,然后回代求解。\par
运算量为 $O(n^3)$\par
可行的充要条件是 $\mathsf{A}$ 的各阶顺序主子式不为零。\par
对角元很小会导致舍入误差的扩散。
\subsubsection{列主元消元法}
将第 $k$ 列第 $k$ 到 $n$ 个元素绝对值最大的元素放在对角位置,然后进行消元。\par
相较于 Gauss 消元法只增加了选列主元和交换两行元素的过程。
\subsubsection{Gauss-Jordan 消元法}
将系数矩阵进一步化为对角矩阵。\par
但是化为对角阵的工作量略大于回代求解。
\subsection{直接分解法}
\subsubsection{Dolittle 分解}
$\mathsf{L}$ 为单位下三角阵(对角元为 $1$ ), $\mathsf{U}$ 为上三角阵。\par
\paragraph{计算步骤}
$\mathsf{U}$ 第一行,$\mathsf{L}$ 第一列,$\mathsf{U}$ 第二行,$\mathsf{L}$ 第二列……
\subsubsection{Courant 分解}
$\mathsf{L}$ 为下三角阵, $\mathsf{U}$ 为单位上三角阵。\par
\paragraph{计算步骤}
$\mathsf{L}$ 第一列, $\mathsf{U}$ 第一行,$\mathsf{L}$ 第二列, $\mathsf{U}$ 第二行……
\subsubsection{追赶法}
\section{线性方程组--迭代法}
\subsection{Jacobi 迭代}

\subsection{Gauss-Seidel 迭代}

\subsection{松弛迭代}

\subsection{逆矩阵计算}

\section{矩阵的特征值与特征向量}
\subsection{幂法}
求解按模最大特征值。
\subsubsection{幂法的计算}
选取初值 $\mathbf{X}^{(0)}$, 构造向量序列
\[
\mathbf{X}^{(k)}=\mathsf{A}\mathbf{X}^{(k-1)}
\]\par
若相邻两个向量各个分量比趋于一个常数,则
\[
\lambda_1\approx x_i^{(k+1)}/x_i^{(k)},\ \ \mathbf{v}_1\approx\mathbf{X}^{(k)}
\]\par
若奇偶序列各个分量比分别趋向于常数,则
\[
\lambda_1\approx\sqrt{x_i^{(k+1)}/x_i^{(k-1)}},\ \ \mathbf{v}_{1,2}\approx\mathbf{X}^{(k+1)}\pm\lambda_1\mathbf{X}^{(k)}
\]
\subsubsection{幂法的规范运算}
避免计算溢出、归零。
\[
\mathbf{Y}^{(k)}=\mathbf{X}^{(k)}/\|\mathbf{X}^{(k)}\|_{\infty},\ \ \mathbf{X}^{(k+1)}=\mathsf{A}\mathbf{Y}^{(k)}\ \ (k=0,1,\cdots)
\]
\subsection{反幂法}
求解按模最小特征值。
\subsection{实对称矩阵的 Jacobi 法}

\subsection{QR 方法}

\section{常微分方程数值解}
\paragraph{基本步骤}
\subparagraph{(1)}对区间做分割
\subparagraph{(2)}建立求格点函数的差分方程(递推关系式):微分、积分、 Taylor 展开
\subparagraph{(3)}解差分方程,求出格点函数
\subsection{Euler 公式}
\subsubsection{基于数值差商}
\paragraph{向前差商}
\[
y_{n+1}=y_n+hf(x_n,y_n)
\]\par
是一步显式格式。
\paragraph{向后差商}
\[
y_{n+1}=y_n+hf(x_{n+1},y_{n+1})
\]\par
是一步隐式格式,需要用迭代求解。\par
Picard 迭代格式\[
y^{(0)}_{n+1}=y_n+hf(x_{n},y_{n}),\ \ y^{(k+1)}_{n+1}=y_n+hf(x_{n+1},y^{(k)}_{n+1})
\]
\paragraph{中心差商}
\[
y_{n+1}=y_{n-1}+2hf(x_n,y_n)
\]\par
是二步二阶格式,因为该格式不稳定,故不采用。
\subsubsection{基于数值积分}
\[
y(x_{n+1})=y(x_n)+\int_{x_n}^{x_{n+1}}f(x,y)\mathrm{d}x
\]
\paragraph{矩形近似}
可得到向前/向后差分公式。
\paragraph{梯形近似}
\[
y(x_{n+1})=y(x_n)+\frac{h}{2}(f(x_n,y_n)+f(x_{n+1},y_{n+1}))
\]\par
此为隐式格式。\par
一般先用其他较为粗糙的显式公式预估初始值,再用隐式公式迭代一次进行修正,称为\textbf{预估-校正过程}。
\paragraph{改进后的 Eular 公式}
\[
y(x_{n+1})=y(x_n)+\frac{h}{2}(f(x_n,y_n)+f(x_{n+1},y_{n}+hf(x_n,y_n)))
\]
\subsection{Runge-Kutta 法}

\subsection{线性多步法}

\subsection{常微分方程组的数值解法}
\end{document}